%=============================================================================
\section{The Equations of Moment-Based RT and RHD with the M1 closure}
%=============================================================================


%----------------------------------------------------
\subsection{The Equations of Radiative Transfer}
%----------------------------------------------------

GEARRT follows the \citet{RAMSES} scheme.


The equation of radiative transfer is given by

\begin{align}
    \frac{1}{c} \DELDT{I_\nu} + \mathbf{n} \cdot \nabla I_\nu 
        &= \eta_\nu - \kappa_\nu I_\nu \label{eq:RT} \\
        &= \eta_\nu - \sum_j^{\absorbers} \sigma_{j,\nu} n_j I_\nu \label{eq:RT-sigma}
\end{align}

$I_\nu$ is called the specific intensity, $\eta_\nu$ is a source function of 
radiation, while $\kappa_\nu$ is an absorption coefficient. 
In eq. \ref{eq:RT-sigma} we split the absorption coefficient $\kappa_\nu$ into the sum over 
all photoabsorbing species $j$ (e.g. hydrogen, helium, and singly ionized helium) 
with their respective interaction cross sections $\sigma_{j,\nu}$ and number 
density $n_j$.

In what follows, we explicitly label the source terms as $\eta_\nu = \dot{E}^*_\nu +
\dot{E}^{rec}_\nu$, where $\dot{E}^*_\nu$ is radiation emitted by stars, and 
$\dot{E}^{rec}_\nu$ is radiation stemming from recombination processes.

Because we don't want to solve the equation for all conceivable directions
$\mathbf{n}$, we take the zeroth and first angular moment of \ref{eq:RT} and
obtain

\begin{align}
	\DELDT{E_\nu} + \nabla \cdot \F_\nu &=
		- \sum\limits_{j}^{\absorbers} n_j \sigma_{\nu j} c E_\nu + \dot{E}_\nu^* + \dot{E}_\nu ^{rec}
		\label{eq:dEdt-freq} \\
	\DELDT{\F_\nu} + c^2 \ \nabla \cdot \mathds{P}_\nu &=
		- \sum\limits_{j}^{\absorbers} n_j \sigma_{\nu j} c \F_\nu
		\label{eq:dFdt-freq}
\end{align}

Note that $E_\nu$, $\dot{E}^*_\nu$, and $\dot{E}^{rec}_\nu$ are radiation energy
\emph{densities} in the frequency interval between frequency $\nu$ and $\nu +
\de \nu$ and have units of $\text{erg / cm}^3 \text{ / Hz}$. 
$\F$ is the radiation flux, and has units of $\text{erg / cm}^2 \text{ / Hz / s}$. 
\footnote{
    It might be helpful to remember that in the optically thin (free streaming)
    limit, the relation between radiation flux and energy density is $|\F| = c
    E$, and in the optically thick (isotropic, LTE) limit, the relation is $|\F|
    = \frac{1}{3}c E$
}


To close this set of equations, an expression (model) for the pressure tensor 
$\mathds{P}_\nu$ is necessary.  We use the so-called ``M1 closure'' where we 
describe the pressure tensor via the Eddington tensor $\mathds{D}_\nu$:

\todo{Citation for M1 closure}

\begin{equation}
	\mathds{P}_\nu = \mathds{D} E_\nu
\end{equation}

The Eddington tensor is a dimensionless quantity that encapsulates the local 
radiation field geometry and its effect in the radiation flux conservation equation.
The M1 closure sets the Eddington tensor to have the form

\begin{align}
	\mathds{D}_\nu &= 
        \frac{1- \chi_\nu}{2} \mathds{I} + \frac{3 \chi_\nu - 1}{2} \mathbf{n}_\nu \otimes \mathbf{n}_\nu \label{eq:eddington-freq} \\
	\mathbf{n}_\nu &= 
        \frac{\F_\nu}{|\F_\nu|} \\
	\chi_\nu &= 
        \frac{3 + 4 f_\nu ^2}{5 + 2 \sqrt{4 - 3 f_\nu^2}} \\
	f_\nu &= 
        \frac{|\F_\nu|}{c E_\nu}
\end{align}









%---------------------------------------------------
\subsection{RHD: Coupling to Hydrodynamics}
%---------------------------------------------------

Currently, we don't take into account the effects of radiation pressure on the
gas. We only consider heating and ionization.

Let 
\begin{equation}
    N_\nu = E_\nu / (h \nu)
\end{equation}
be the local photon \emph{number} density.

The photoheating rate of the gas is given by
\begin{align}
\mathcal{H}_{\nu, j} = (h \nu - h \nu_{ion,j}) c \ \sigma_{\nu j} \ n_j \ N_\nu
\end{align}

where $\nu_{ion,j}$ is the ionization energy of the species $j$. In the case of
hydrogen and helium, their values are

\begin{align}
    \nu_{ion,HI} &= 2.179 \times 10^{-11} \text{ erg} = 13.60 \text{ eV}\\
    \nu_{ion,HeI} &= 3.940 \times 10^{-11} \text{ erg} = 24.59 \text{ eV}\\
    \nu_{ion,HeII} &= 8.719 \times 10^{-11} \text{ erg} = 54.42 \text{ eV}
\end{align}

and the photoionization rate of the gas is given by
\begin{align}
\Gamma_{\nu, j} = c \ \sigma_{\nu j} \ N_\nu
\end{align}
