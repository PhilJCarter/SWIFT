%--------------------------------------------
\section{Function Calls of an RT step}
%--------------------------------------------



\begin{landscape}
{\footnotesize

\begin{tabular}[l]{%
	>{\raggedright\arraybackslash}p{2.6cm}%
	>{\raggedright\arraybackslash}p{2.8cm}%
	>{\raggedright\arraybackslash}p{7cm}%
	>{\raggedright\arraybackslash}p{7cm}%
}
\textbf{task} & \textbf{task type} & \textbf{task purpose} & \textbf{function calls} \\[.5em]
\hline
\hline
Injection Prep &
	Interacting star and gas particles &
	Gather gas particle neighbour data in preparation for the injection &
	\texttt{runner\_iact\_rt\_inject\_prep} in \verb|src/rt/method/rt_iact.h| \\
\hline
Star Emission Rates &
	Work on individual star particles &
	Prepare everything necessary that needs to be done for radiation sources before the radiation sources and the gas interact. &
	\texttt{rt\_compute\_stellar\_emission\_rate} in \verb|src/rt/method/rt.h| \\
\hline
\hline
RT in &
	Implicit&
  Collect dependencies &
	- \\
\hline
Injection &
	Interacting star and gas particles &
	Distribute the radiation from star particles onto gas particles &
	\verb|runner_iact_rt_inject| in \verb|src/rt/method/rt_iact.h| \\
\hline
Ghost1 &
	Work on individual gas particles &
	Finish up any work that needs to be done for gas particles before the next gas $\leftrightarrow$ gas interaction begins &
	\texttt{rt\_finalise\_injection} in \verb|src/rt/method/rt.h|\\
\hline
Gradient &
	Interacting gas and gas particles &
	Compute necessary gradients of the radiation quantities &
	\verb|rt_gradients_collect| in \verb|src/rt/method/rt_gradients.h| and
	\verb|rt_gradients_nonsym_collect| in \verb|src/rt/method/rt_gradients.h|\\
\hline
Ghost2 &
	Work on individual gas particles &
	Finish up any work that needs to be done for gas particles before the next gas $\leftrightarrow$ gas interaction begins &
	\texttt{rt\_end\_gradient} in \verb|src/rt/method/rt.h|\\
\hline
Transport &
	Interacting gas and gas particles &
	Compute/exchange fluxes of homogenized equation of radiative transfer. &
	\verb|runner_iact_rt_transport| in \verb|src/rt/method/rt_iact.h| and
	\verb|runner_iact_nonsym_rt_transport| in \verb|src/rt/method/rt_iact.h|\\
\hline
Transport out &
	Implicit&
  Collect dependencies &
	- \\
\hline
Thermochemistry &
	Work on individual gas particles &
	Finish up any work that needs to be done for gas particles before the thermochemistry part of the computation can begin. Then do the thermochemistry computation. &
	\texttt{rt\_finalise\_transport} in \verb|src/rt/method/rt.h|,
	\verb|rt_do_thermochemistry| in \verb|src/rt/method/rt_thermochemistry.h|\\
\hline
RT out &
	Implicit&
  Collect dependencies &
	- \\
\hline
\end{tabular}


}
\end{landscape}
\newpage







%-------------------------------------------------------------------------
\section{Creating Collisional Ionization Equilibrium Initial Conditions}
%-------------------------------------------------------------------------


On startup, GEARRT offers the possibility to generate the ionization mass fractions of the gas 
particles assuming the gas is in collisional ionization equilibrium, composed of Hydrogen and 
Helium, and that there is no radiation present. In order to determine the ionization mass fractions 
of all species (H$^0$, H$^+$, He$^0$, He$^+$, He$^{++}$) for a given specific internal energy $u$, 
an iterative procedure needs to be applied because the gas variables are interconnected:



Commonly the average particle mass $\overline{m}$ of a gas is expressed using the (unitless) mean 
molecular weight $\mu$ : 

\begin{align}
  \overline{m} = \mu m_u
\end{align}

where $m_u$ is the atomic mass unit.

The ionization state of the gas influences the mean molecular weight $\mu$. Consider a medium with 
$j$ different elements of atomic mass $A_j$ with a mass fraction $X_j$ and a number of free 
electrons $E_j$. Then the number density of the gas is given by

\begin{align}
    n = \frac{\rho}{\mu m_u} = 
        \sum_j \underbrace{\rho X_j}_{\text{density fraction of species } j} \times
        \underbrace{\frac{1}{A_j m_u}}_{\text{mass per particle of species}} \times
        \underbrace{(1 + E_j)}_{\text{number of particles per species}} 
\label{eq:number-density-MMW}
\end{align}


whre we neglect the mass contribution of electrons. In the case for Hydrogen and Helium, the values 
of $A_j$ and $E_j$ are shown in Table \ref{tab:mass-and-electron-numbers}. Eq. 
\ref{eq:number-density-MMW} simplifies to

\begin{align}
    \frac{1}{\mu} = \sum_j \frac{X_j}{A_j} (1 + E_j)
\end{align}

Specifically, ionization changes the mass fractions $X_j$ of the species, and therefore also the 
mean molecular weight $\mu$.

In turn, the mean molecular weight determines the gas temperature at a given specific internal 
energy. Using the ideal gas law

\begin{align}
    p = n k T = \frac{\rho}{\mu m_u} k T
\end{align}

and the expression for the specific internal energy

\begin{align}
    u = \frac{1}{\gamma - 1} \frac{p}{\rho}
\end{align}

the gas temperature $T$ = $T(u, \mu)$ is given by

\begin{align}
    T = u (\gamma - 1) \mu \frac{m_u}{k}
\end{align}`

Lastly, the gas temperature determines the collisional ionization and recombination rates, which 
need to be balanced out by the correct number density of the individual species in order to be in 
ionization equilibrium, i.e. at a state where the ionization and recombination rates exactly cancel 
each other out.

We take the ionization and recombination rates from \citet{katzCosmologicalSimulationsTreeSPH1996}, 
which are given in Table \ref{tab:coll-ion-rates-katz}. For a gas with density $\rho$, Hydrogen
mass fraction $X_H$ and Helium mass fraction $X_{He} = 1 - X_H$, the total number densities of all 
hydrogen and helium species are

\begin{align}
  n_H &= X_H \frac{\rho}{m_u} \\
  n_{He} &= X_{He}  \frac{\rho}{4 m_u} 
\end{align}

an in equilibrium, the number densities of the individual species are given by

\begin{align}
    n_{H^0} &= 
        n_H \frac{A_{H^+}}{A_{H^+} + \Gamma_{H^0}} \\
    n_{H^+} &= 
        n_H - n_{H^0} \\
    n_{He^+} &= 
        n_{He} \frac{1}{1 + (A_{He^+} + A_d) / \Gamma_{He^0} + \Gamma_{He^+} / A_{He^{++}}} \\
    n_{He^0} &= 
        n_{He^+} \frac{A_{He^+} + A_d}{\Gamma_{He^0}} \\
    n_{He^{++}} &= 
        n_{He^+} \frac{\Gamma_{He^+}}{A_{He^+}}
\end{align}


To summarize, the tricky bit here is that the number densities determine the mean molecular weight, 
the mean molecular weight determines the temperature of the gas for a given density and specific 
internal energy, while the temperature determines the number densities of the species.

To find the correct mass fractions, the iterative Newton-Raphson root finding method is used. 
Specifically, using some initial guesses for temperature and mean molecular weights, $T_{guess}$ 
and $\mu_{guess}$, in each iteration step we determine the resulting specific internal energy 
$u_{guess} = k T_{guess} / (\gamma - 1) / (\mu_{guess} m_u)$. The function whose root we're looking 
for is

\begin{align}
  f(T) = u - u_{guess}(T)
\end{align}

with the derivative

\begin{align}
  \frac{\del f}{\del T} =  - \frac{\del u}{\del T} (T = T_{guess}) = \frac{k}{(\gamma - 
1) / (\mu_{guess} m_u )}
\end{align}

where $u$ is the specific internal energy of the gas, which is fixed and provided by the initial 
conditions. We now look for the $T$ at which $f(T) = 0$. The Newton-Raphson method prescribes to 
find the $n+1$th $T_{guess}$ using

\begin{align}
    T_{n+1} = T_n + \frac{f(T_n)}{\frac{\del f}{\del T}(T_n)}
\end{align}

During each iteration, the new mass fractions and the resulting mean molecular weight given the 
latest guess for the temperature are computed. At the start, the first guess for the temperature 
$T_{guess}$ is computed assuming a fully neutral gas. Should that gas temperature be above $10^5$ K, 
the first guess is changed to a fully ionized gas. The iteration is concluded once $f(T) \leq 
\epsilon = 10^{-4}$.



\begin{table}
\begin{center}
\begin{tabular}{lll}
\hline
$A_{H^+}         = $ & 
    $8.4 \times 10^{-11} T^{-1/2} T_3^{-0.2} (1 + T_6^{0.7})^{-1}$ & 
    RR for H$^{+}$ \\
$A_{d}           = $ &
    $1.5 \times 10^{-10} T^{-0.6353}$    & 
    dielectronic RR for He$^{+}$ \\
$A_{He^+}        = $ & 
    $1.9 \times 10^{-3} T^{-1.5} \mathrm{e}^{-470000/T} (1 + 0.3 \mathrm{e}^{-94000/T})$    & 
    RR for He$^{+}$ \\
$A_{He^{++}}     = $ & 
    $3.36 \times 10^{-10} T^{-1/2} T_3^{-0.2} (1 + T_6^{0.7})^{-1}$    & 
    RR for He$^{++}$ \\
\hline
$\Gamma_{H^{0}}  = $ & 
    $5.85 \times 10^{-11} T^{1/2} \mathrm{e}^{-157809.1/T} ( 1 + T_5^{1/2})^{-1}$& 
    CIR for H$^{0}$ \\
$\Gamma_{He^{0}} = $ & 
    $2.38 \times 10^{-11} T^{1/2} \mathrm{e}^{-285335.4.1/T} ( 1 + T_5^{1/2})^{-1}$& 
    CIR for He$^{0}$ \\
$\Gamma_{He^{+}} = $ & 
    $5.68 \times 10^{-12} T^{1/2} \mathrm{e}^{-631515/T} ( 1 + T_5^{1/2})^{-1}$& 
    CIR for He$^{+}$ \\
\hline
\end{tabular}
\end{center}
\caption{Temperature ($T$) dependent recombination rates (RR) and collisional ionization rates 
(CIR) for Hydrogen and Helium species, adapted from \citet{katzCosmologicalSimulationsTreeSPH1996}. 
All rates are in units of cm$^3$ s$^{-1}$. $T_n$ is shorthand for $T / 10^n$K.}
\label{tab:coll-ion-rates-katz}
\end{table}










\begin{table}
\begin{center}
\begin{tabular}{l|ll}
species  & $A$ & $E$ \\
\hline
H$^0$    & 1 & 0 \\
H$^+$    & 1 & 1 \\
He$^0$   & 4 & 0 \\
He$^+$   & 4 & 1 \\
He$^{++}$  & 4 & 2 \\
\end{tabular}
\end{center}
\caption{Atomic mass numbers $A$ and free electron numbers $E$ for ionization species of Hydrogen 
and Helium.}
\label{tab:mass-and-electron-numbers}
\end{table}










%-------------------------------------------------------------------------------
\section{Converting Photon Number Emission Rates to Photon Energy Emission Rates}
%-------------------------------------------------------------------------------

Many other, in particular older, codes and papers use photon \emph{number} injection rates $\dot{N}_{\gamma}$ for their emission rates rather than the \emph{energy} injection rates $\dot{E}_\gamma$ or equivalently luminosities $L$.

To convert between these two quantities, we need to assume that the emission follows some spectrum $J(\nu)$.

In the case of a single photon group, the conversion is quite simple: We first need to compute the average photon energy $\overline{E}_\gamma$:

\begin{align}
	\overline{E}_\gamma = \frac{\int J(\nu) \ \de \nu }{\int J(\nu) / (h \nu) \ \de \nu}
\end{align}

then the emitted luminosity (energy per unit time) is

\begin{align}
	L = \overline{E}_\gamma \ \dot{N}_{\gamma}
\end{align}

Note that in many cases, the given emission photon number rate is the number rate of \emph{ionizing} photons. For us, this means that we need to start the integrals at the lowest ionizing frequency $\nu_{\text{ion, min}}$ in order to have the correct translation to the luminosity of the \emph{ionizing} energy:


\begin{align}
	\overline{E}_\gamma = \frac{\int\limits_{\nu_{\text{ion, min}}}^\infty J(\nu) \ \de \nu }{\int\limits_{\nu_{\text{ion, min}}}^\infty J(\nu) / (h \nu) \ \de \nu}
\end{align}


In the case of several photon groups being used, the conversion requires a little adaptation in order to preserve the correct number of photons emitted. For each photon group $i$, the average photon energy is given by

\begin{align}
	\overline{E}_i = \frac{\int\limits_{\nu_{i \text{, min}}}^{\nu_{i \text{, max}}} J(\nu) \ \de \nu }{\int\limits_{\nu_{i \text{, min}}}^{\nu_{i \text{, max}}} J(\nu) / (h \nu) \ \de \nu}
\end{align}


Secondly, we need to compute the fraction $f_i$ of ionizing photons in each bin, which is given by

\begin{align}
	f_i = \frac{\int\limits_{\nu_{i \text{, min}}}^{\nu_{i \text{, max}}} J(\nu) / (h \nu)  \ \de \nu }{\int\limits_{\nu_{min}}^{\infty} J(\nu) / (h \nu) \ \de \nu}
\end{align}

Then the number of emitted photons in each bin is given by

\begin{align}
\dot{N}_i = f_i\ \dot{N}_\gamma
\end{align}

And the luminosities are given by

\begin{align}
	L_i &= \overline{E}_i \ \dot{N}_i \\
			&= \frac{\int\limits_{\nu_{i \text{, min}}}^{\nu_{i \text{, max}}} J(\nu) \ \de \nu }{\int\limits_{\nu_{min}}^{\infty} J(\nu) / (h \nu) \ \de \nu} \ \dot{N}_\gamma
\end{align}




%---------------------------------------------
\section{Additional Notes}
%---------------------------------------------

%---------------------------------------------------------------------------------
\subsection{Zero Flux with Nonzero Energy} \label{chap:zero-flux-nonzer-energy}
%---------------------------------------------------------------------------------

We could encounter cases where we have nonzero radiation energy, but zero
radiation flux. (E.g. through diffusion or when exception handling unphysical
scenarios.) For these cases, recall that (index $i$ below is for photon group,
not particle index!)

\begin{align}
	\DELDT{\F_i} + c^2 \ \nabla \cdot \mathds{P}_i &=
		- \sum\limits_{j}^{\mathrm{HI, HeI, HeII}} n_j \sigma_{i j} c \F_i \\
	\mathds{P}_i &=
		\mathds{D}_i E_i \\
	\mathds{D}_i &=
		\frac{1- \chi_i}{2} \mathds{I} + \frac{3 \chi_i - 1}{2} \mathbf{n}_i \otimes \mathbf{n}_i\\
	\mathbf{n}_i &=
		\frac{\F_i}{|\F_i|} \\
	\chi_i &=
		\frac{3 + 4 f_i ^2}{5 + 2 \sqrt{4 - 3 f_i^2}} \\
	f_i &=
		\frac{|\F_i|}{c E_i}
\end{align}

For $\F_i = 0$, we get
\begin{align}
	f_i &= 0 \\
	\chi_i &= \frac{3}{5 + 2 \sqrt{4}} = \frac{1}{3} \\
	\mathds{D}_i &= \frac{1- \frac{1}{3}}{2} \mathds{I} = \frac{1}{3} \mathds{I} \\
	\mathds{P}_i &= \mathds{D}_i E_i =  \frac{1}{3} E_i \mathds{I}
\end{align}

which is the solution of the optically thick limit, where the radiation pressure tensor is 
isotropic.


Now let's look at what happens when a particle $k$ has some nonzero energy $E_0$, but zero flux, 
and interacts with particle $l$, which has both zero energy density and zero flux. The initial 
state is then:

\begin{align}
    \mathcal{U}_k(t = 0) &= \left( \begin{matrix}
                    E_0 \\
                    \mathbf{0}
                  \end{matrix} \right)
    \quad & \quad \quad
    \mathcal{F}_k(t = 0)  &= \left( \begin{matrix}
                   \mathbf{0} \\
                   c^2 \mathds{P}
                  \end{matrix} \right)
                = \left( \begin{matrix}
                   \mathbf{0} \\
                   \frac{c^2}{3} E_0 \mathds{I}
                  \end{matrix} \right)\\
%
    \mathcal{U}_l(t = 0) &= \left( \begin{matrix}
                    0 \\
                    \mathbf{0}
                  \end{matrix} \right)
    \quad & \quad \quad
    \mathcal{F}_l(t = 0)  &= \left( \begin{matrix}
                   \mathbf{0} \\
                   0
                  \end{matrix} \right)
\end{align}

To simplify matters, let's assume both particles have equal volumes, $V_k = V_l = V$, and let's 
omit the gradient extrapolation to the interface position that makes the method second order 
accurate. Let's also assume that the particles are aligned parallel to a coordinate axis,  thus 
making the projection along the normal vector to their interaction surface trivial.

Then the intercell flux given by the GLF Riemann solver (eq. \ref{eq:riemann-GLF}) is

\begin{align}
	\mathcal{F}_{1/2}(\mathcal{U}_L, \mathcal{U}_R) &=
		\frac{\mathcal{F}_{L} + \mathcal{F}_{R}}{2} -
		\frac{c}{2} \left(\mathcal{U}_R - \mathcal{U}_L \right) \\
	&=	\left(\begin{matrix}
        \frac{0 - 0}{2} - \frac{c}{2} (0 - E_0)\\
        \frac{c^2 / 3 \ E_0 - 0}{2} - \frac{c}{2} (0 -0)
	  	\end{matrix} \right)
	=	\left(\begin{matrix}
        \frac{c}{2} E_0 \\
        \frac{c^2}{6} E_0
	  	\end{matrix} \right)
\end{align}

Let $\beta \equiv \frac{\Delta t A_{kl} c}{2 V}$. Using the update formula given in eq. 
\ref{eq:transport-update-explicit}, the states at $t = \Delta t$ will be

\begin{align}
	\mathcal{U}_k(t = \Delta t) &= \mathcal{U}_k (t = 0) - \frac{\Delta t}{V} \mathcal{F}_{1/2}
A_{kl} \\
	&=	\left(\begin{matrix}
        E_0 -  \frac{\Delta t A_{kl}}{V} \frac{c}{2} E_0 \\
        \mathbf{0} - \frac{\Delta t A_{kl}}{V} \frac{c^2}{6} E_0
	  	\end{matrix} \right)
    = \left( \begin{matrix}
             E_0 (1 - \beta) \\
             -\frac{\beta}{3} c E_0
             \end{matrix} \right) \\
%
	\mathcal{U}_l(t = \Delta t) &= \mathcal{U}_l (t = 0) - \frac{\Delta t}{V} (-\mathcal{F}_{1/2}) 
A_{kl} \label{eq:zero-flux-update-l}\\
	&=	\left(\begin{matrix}
        0 +  \frac{\Delta t A_{kl}}{V} \frac{c}{2} E_0 \\
        \mathbf{0} + \frac{\Delta t A_{kl}}{V} \frac{c^2}{6} E_0
	  	\end{matrix} \right)
    = \left( \begin{matrix}
             \beta E_0 \\
             \frac{\beta}{3} c E_0
             \end{matrix} \right)
\end{align}

where the minus sign for the intercell flux $(-\mathcal{F}_{1/2})$ in eq. 
\ref{eq:zero-flux-update-l} stems from the orientation of the effective surface $\mathbf{A}_{kl}$.


The takeaway here is that after a single time step, both particles $k$ and $l$ will have nonzero 
energy \emph{and} nonzero photon flux $\F$. 
For the photon fluxes of the particle $l$, we see that $|\F_l| = \frac{1}{3} c E_l$ with $E_l = 
\beta E_0$, meaning that the photon flux corresponds to the optically thick, diffusion limit. For 
particle $k$ we have $|\F_k| = \frac{\beta}{1 - \beta} c E_k$ with $E_k = (1 - \beta) E_0$. 
Comparing the particle volume $V$ and effective surface $A_{kl}$ to a regular cell of edge length 
$L$, then $V = L^3$ and $A_{kl} = L^2$. Simultaneously this gives us a comparative CFL condition: $L 
\geq c \Delta t$. This simple comparison limits the possible values of $\beta$ to $0 \leq \beta 
\leq \frac{1}{2}$, given that all terms of $\beta$ are be positive. The limits translate to 
$0 \leq |\F_k| \leq c E_k$ for the photon flux, meaning that particle $k$ can end up in any state 
between no net flux, and the free streaming, optically thin limit.









%------------------------------------------------
\subsection{RT Related Quantities And Their Units}
%------------------------------------------------

It could be helpful to have the commonly used quantities of RT written down
somewhere along with their units. So here you go.


\begin{align*}
	I_\nu (\x, \mathbf{n}, t) &
			&& \text{Specific intensity}
			&& [I_\nu] = \frac{\text{erg}}{\text{cm}^2 \text{ rad Hz s}} \\
	u_\nu (\x, \mathbf{n}, t) &= \frac{I_\nu}{c}
			&& \text{radiation energy density }
			&& [u_\nu] = \frac{\text{erg}}{\text{cm}^3 \text{ rad Hz}}\\
	E_\nu (\x, t) &= \int_{4 \pi} \frac{I_\nu}{c} \de \Omega
			&& \text{total energy density }
			&& [E_\nu] = \frac{\text{erg}}{\text{cm}^3 \text{ Hz}}\\
	E_{rad} (\x, t) &= \int_{0}^{\infty} E_\nu \de \nu
			&& \text{total integrated energy density }
			&& [E_\nu] = \frac{\text{erg}}{\text{cm}^3}\\
	J_\nu (\x, t) &= \int_{4 \pi} I_\nu \frac{\de \Omega}{4 \pi}
			= \frac{c}{4 \pi} E_\nu
			&& \text{mean radiation specific intensity }
			&& [J_\nu] = \frac{\text{erg}}{\text{cm}^2 \text{ Hz s}}\\
	\F_\nu(\x, t) &= \int_{4 \pi}  I_\nu \mathbf{n} \de \Omega
			&& \text{radiation flux }
			&& [\F_\nu] = \frac{\text{erg}}{\text{cm}^2 \text{ s Hz}}\\
	\mathbf{P}_\nu (\x, t) &= \frac{\F_\nu}{c^2}
			&& \text{radiation momentum density }
			&& [\mathbf{P}_\nu] = \frac{\text{erg}}{ \text{cm}^4 \text{ s}^{-1} \text{ Hz}}\\
	\mathds{P}_\nu (\x, t) &= \int_{4 \pi} \frac{I_\nu}{c} \mathbf{n} \otimes \mathbf{n} \de \Omega
			&& \text{radiation pressure tensor }
			&& [\mathds{P}_\nu ] = \frac{\text{erg}}{\text{cm}^3 \text{ Hz}}
\end{align*}



